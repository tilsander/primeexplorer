% 
% Copyright (c) 2014 Til Sander
% 
% Permission is hereby granted, free of charge, to any person obtaining a copy
% of this software and associated documentation files (the "Software"), to deal
% in the Software without restriction, including without limitation the rights
% to use, copy, modify, merge, publish, distribute, sublicense, and/or sell
% copies of the Software, and to permit persons to whom the Software is
% furnished to do so, subject to the following conditions:
% 
% The above copyright notice and this permission notice shall be included in
% all copies or substantial portions of the Software.
% 
% THE SOFTWARE IS PROVIDED "AS IS", WITHOUT WARRANTY OF ANY KIND, EXPRESS OR
% IMPLIED, INCLUDING BUT NOT LIMITED TO THE WARRANTIES OF MERCHANTABILITY,
% FITNESS FOR A PARTICULAR PURPOSE AND NONINFRINGEMENT. IN NO EVENT SHALL THE
% AUTHORS OR COPYRIGHT HOLDERS BE LIABLE FOR ANY CLAIM, DAMAGES OR OTHER
% LIABILITY, WHETHER IN AN ACTION OF CONTRACT, TORT OR OTHERWISE, ARISING FROM,
% OUT OF OR IN CONNECTION WITH THE SOFTWARE OR THE USE OR OTHER DEALINGS IN
% THE SOFTWARE.

\documentclass[12pt,ngerman]{article}
\usepackage{lmodern}
\usepackage{helvet}
\usepackage[T1]{fontenc}
\usepackage[utf8]{inputenc}
\usepackage{geometry}
\geometry{verbose,lmargin=2cm,rmargin=1.7cm,headheight=0cm}
\usepackage{xcolor}
\usepackage{listings}

\usepackage{amsmath}
\usepackage{amssymb}
\usepackage{amsthm}
\usepackage{fixltx2e}
\usepackage{babel}

\usepackage{alltt}
\usepackage{setspace}
\usepackage[rounded]{syntax}
\usepackage{diagbox}
\usepackage{tikz}
\usetikzlibrary{arrows}
\usetikzlibrary{shapes}
\usetikzlibrary{trees}

            
\newcommand{\trmnl}[1]{%
  \tikz[baseline=(char.base)]\node[draw, fill=white, shape=rounded rectangle, minimum height=12pt](char){\ensuremath{#1}} ;}
            
\newcommand{\nontrmnl}[1]{\tikz[baseline=(char.base)]\node[draw, fill=white, shape=rectangle, minimum height=12pt](char){#1} ;}

\newcommand{\lr}[1]{$\langle$\texttt{{#1}}$\rangle$}
\newcommand{\lru}[1]{$\langle$\texttt{\underline{{#1}}}$\rangle$}
\newcommand{\ra}{$\rightarrow$\ }

\lstset{
  language=Java,
  basicstyle=\footnotesize\ttfamily,
  breakatwhitespace=true,
  breaklines=true,
  keywordstyle=\color{corange}\ttfamily,
  keywordstyle=[2]\color{cpurple}\ttfamily,
  keywordstyle=[3]\color{cblue}\ttfamily,
  keywordstyle=[4]\color{clightblue}\ttfamily,
  stringstyle=\color{cred}\ttfamily,
  identifierstyle=\color{cdarkgreen}\ttfamily,
  numbers=left,
  numberstyle=\color{clightgray}\scriptsize,
  stepnumber=1,
  numbersep=8pt,
  commentstyle=\color{cbrown}\ttfamily,
  morekeywords={if},
  keywords=[2]{define,define-struct},
  keywords=[3]{else},
  keywords=[4]{true,false},
  morecomment=[l][\color{cbrown}]{;;},
  backgroundcolor=\color{cwhite}
}

\definecolor{cblue}{RGB}{20,105,176}
\definecolor{clightblue}{RGB}{60,175,236}
\definecolor{cgreen}{RGB}{100,185,56}
\definecolor{cred}{RGB}{240,15,76}
\definecolor{corange}{RGB}{210,116,56}
\definecolor{cpurple}{RGB}{230,95,176}
\definecolor{cbrown}{RGB}{150,150,125}
\definecolor{cdarkgray}{RGB}{85,85,85}
\definecolor{clightgray}{RGB}{204,204,204}
\definecolor{cwhite}{RGB}{245,245,245}
\definecolor{cdarkgreen}{RGB}{0,100,0}
\definecolor{cdarkblue}{RGB}{20,40,160}

\def\digitstyle{\color{cdarkblue}\ttfamily}
\makeatletter


\def\SPSB#1#2{\rlap{\textsuperscript{\textcolor{darkgray}{#1}}}\SB{#2}}
\def\SP#1{\textsuperscript{\textcolor{darkgray}{#1}}}
\def\SB#1{\textsubscript{\textcolor{darkgray}{#1}}}
\def\SC#1{\lstinline$#1$}
\def\DG#1{\textcolor{cdarkgreen}{#1}}

\makeatletter

\providecommand{\tabularnewline}{\\}

\theoremstyle{definition}
\newtheorem{defninn}{Definition}

\newcounter{defcounter}

\makeatletter
\newenvironment{defn}
 {\global\chardef\dc@currentequation=\value{equation}%
  \let\c@equation\c@defcounter
  \renewcommand{\theequation}{\arabic{equation}}%
  % comment the following line if you don't use hyperref
  % \renewcommand{\theHequation}{D\arabic{equation}}%
  \defninn}
 {\enddefninn
  \setcounter{equation}{\dc@currentequation}}

\makeatother
\begin{document}
The prime-counting function:\\\\
The sieve of Eratosthenes is an easy method to generate primes.\\
You simply start at 2, and mark the multiples as composite numbers.\\
This is repeated with the next number, which was not marked as a composite, i.e. the next prime.\\
With this algorithm one can specify the prime-counting function as:\\\\
$\pi(n) = \vert \{2,...,n\} \setminus \bigcup\limits_{i = 2}^{n-1} \bigcup\limits_{k = 2}^{n+1-i} i \cdot k \vert$\\\\
If you shift the indices, the set of composites, which is substracted from the set $\{2,...,n\}$ could also be written as:\\\\
$\bigcup\limits_{i = 2}^{n-1} \bigcup\limits_{k = 2}^{n+1-i} i \cdot k = \bigcup\limits_{i = 3}^{n} \bigcup\limits_{k = 2}^{n+2-i} (i-1) \cdot k = \bigcup\limits_{i = 3}^{n} \bigcup\limits_{k = 2}^{i-1} (n+2-i) \cdot k =: \varphi(n)$\\\\\\
Now consider the next function:\\\\
$\bigcup\limits_{i = 3}^{n} \bigcup\limits_{k = 2}^{i-1} (i+1-k) \cdot k =: \psi(n)$\\\\
The claim is: $\varphi(n) = \psi(n)$ $\forall n \geq 2$, i.e. the two functions generate the same sets.\\\\
The following tables show the development of the sets of composites for $\varphi(n)$ and $\psi(n)$, starting at $n=3$:\\
\begin{table}[h]
\parbox{.45\linewidth}{
\centering
$\varphi(n)$\\
\begin{tabular}{|c|c|c|c|c|c|}
\hline
\diagbox{k}{i}&3&4&5&6&7\\
\hline
2	& \DG{4} & 	& 	& 	& 	\\
\hline
\hline
2	& \DG{6} & 4 & 	& 	&	\\
\hline
3	& 	& \DG{6} & 	& 	&	\\
\hline
\hline
2	& \DG{8} & 6 & 4 & 	&	\\
\hline
3	& 	& \DG{9} & 6 &	&	\\
\hline
4	&	&	& \DG{8} &	&	\\
\hline
\hline
2	& \DG{10}& 8 & 6 & 4 &	\\
\hline
3	&	& \DG{12}& 9 & 6 &	\\
\hline
4	&	&	& \DG{12}& 8 &	\\
\hline
5	&	&	&	& \DG{10}&	\\
\hline
\hline
2	& \DG{12}& 10& 8 & 6 & 4 \\
\hline
3	&	& \DG{15}& 12& 9 & 6 \\
\hline
4 	&	&	& \DG{16}& 12& 8 \\
\hline
5	&	&	&	& \DG{15}& 10\\
\hline
6	&	&	&	&	& \DG{12}\\
\hline
\end{tabular}
}
\hfill
\parbox{.45\linewidth}{
\centering
$\psi(n)$\\
\begin{tabular}{|c|c|c|c|c|c|}
\hline
\diagbox{k}{i}&3&4&5&6&7\\
\hline
2	& \DG{4} & 	& 	& 	& 	\\
\hline
\hline
2	& 4 & \DG{6} & 	& 	&	\\
\hline
3	& 	& \DG{6} & 	& 	&	\\
\hline
\hline
2	& 4 & 6 & \DG{8} & 	&	\\
\hline
3	& 	& 6 & \DG{9} &	&	\\
\hline
4	&	&	& \DG{8} &	&	\\
\hline
\hline
2	& 4 & 6 & 8 & \DG{10}&	\\
\hline
3	&	& 6 & 9 & \DG{12}&	\\
\hline
4	&	&	& 8 & \DG{12}&	\\
\hline
5	&	&	&	& \DG{10}&	\\
\hline
\hline
2	& 4 & 6 & 8 & 10& \DG{12}\\
\hline
3	&	& 6 & 9 & 12& \DG{15}\\
\hline
4 	&	&	& 8 & 12& \DG{16}\\
\hline
5	&	&	&	& 10& \DG{15}\\
\hline
6	&	&	&	&	& \DG{12}\\
\hline
\end{tabular}
}
\end{table}

Proof by Mathematical Induction:\\\\
Base Case: $n=3$\\\\
$\varphi(3) = \bigcup\limits_{i = 3}^{3} \bigcup\limits_{k = 2}^{i-1} (n+2-i) \cdot k = \{4\} = \bigcup\limits_{i = 3}^{3} \bigcup\limits_{k = 2}^{i-1} (i+1-k) \cdot k = \psi(3)$\\\\
Assumption:\\\\
$\varphi(n-1) = \bigcup\limits_{i = 3}^{n-1} \bigcup\limits_{k = 2}^{i-1} (n+1-i) \cdot k = \bigcup\limits_{i = 3}^{n-1} \bigcup\limits_{k = 2}^{i-1} (i+1-k) \cdot k = \psi(n-1)$\\\\
Induction Step: $n-1 \rightarrow n$\\\\
It shows that\\\\
$\varphi(n) = \bigcup\limits_{i = 3}^{n} \bigcup\limits_{k = 2}^{i-1} (n+2-i) \cdot k = \varphi(n-1) \cup \bigcup\limits_{i = 3}^{n} (n+2-i) \cdot (i-1) := \varphi(n-1) \cup \varphi'$\\
and\\\\
$\psi(n) = \bigcup\limits_{i = 3}^{n} \bigcup\limits_{k = 2}^{i-1} (i+1-k) \cdot k = \psi(n-1) \cup \bigcup\limits_{k = 2}^{n-1} (n+1-k) \cdot k := \psi(n-1) \cup \psi'$\\\\
Now we show that the added sets are the equal:\\\\
$\varphi' = \bigcup\limits_{i = 3}^{n} (n+2-i) \cdot (i-1) = \bigcup\limits_{i = 2}^{n-1} (n+2-(i+1)) \cdot ((i+1)-1) = \bigcup\limits_{i = 2}^{n-1} (n+1-i) \cdot i = \bigcup\limits_{k = 2}^{n-1} (n+1-k) \cdot k = \psi'$\\
$\Rightarrow \varphi(n) = \psi(n)$\\
$\Box$\\\\
Applications:\\\\
The prime-counting function:\\
$\pi(n) = \vert \{2,...,n\} \setminus \psi(n) \vert$\\
The following function calculates the number of Goldbach partitions for the even number $2 \cdot n$:\\
$G(n) = \vert \{2,...,n\} \setminus \bigcup\limits_{b = 0}^{1} \bigcup\limits_{i = 3}^{n} \bigcup\limits_{k = 2}^{i-1} n + (-1)^b \cdot (n - (i + 1 - k) \cdot k) \vert$\\\\
The function $\psi(n)$ could be further generalized to use a constanct factor $c \in \mathbb{N}$:\\
$\psi_c(n) = \bigcup\limits_{j = 2-c}^{1} \bigcup\limits_{i = 3}^{n} \bigcup\limits_{k = 2}^{i-1} c \cdot (i + \frac{j}{c} - k) \cdot k$\\\\
$\pi_c(n) = \vert \{2,...,n\} \setminus \psi_c(n) \vert$\\\\
$G_c(n) = \vert \{2,...,n\} \setminus \bigcup\limits_{b = 0}^{1} \bigcup\limits_{j = 2-c}^{1} \bigcup\limits_{i = 3}^{n} \bigcup\limits_{k = 2}^{i-1} n + (-1)^b \cdot (n - c \cdot (i + \frac{j}{c} - k) \cdot k) \vert$\\\\\\
It holds that:\\
$\pi(n) = \pi_c(n) \forall_{n,c \in \mathbb{N}}$\\
$G(n) = G_c(n) \forall_{n,c \in \mathbb{N}}$

\end{document}
